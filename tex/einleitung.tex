\chapter{Einleitung}
\label{chap:einleitung}

Der Apriori Algorithmus ist ein, im IBM-Almaden-Forschungszentrum entwickeltes, klassisches Verfahren der Assoziationsanalyse. Er findet sinnvolle oder nützliche Zusammenhänge in transaktionsbasierten Datenbanken und bildet diese in Assoziationsregeln ab. Ein häufiger Anwendungsfall wäre die Analyse des Kaufverhaltens von Kund\_innen bzw. die Warenkorbanalyse.
\parencite [s.][S. 109, S.112] {Datawarehouse}


\paragraph{Assoziationsanalyse}
Assoziationsanalyse ist ein Bereich des Data Mining und bezeichnet die Suche nach 'starken' Regeln. Daraus folgen Assoziationsregeln, die Korrelationen zwischen gemeinsam auftretenden Ereignissen beschreiben.

\paragraph{Assoziationsregeln}
Assoziationsregeln beschreiben Korrelationen zwischen zwei Ereignissen (mit einer bestimmten Wahrscheinlichkeit) in der Form $ A \rightarrow B $ .\\
Um solche Zusammenhänge zu erkennen werden Datenbestände bezüglich der Häufigkeit des gleichzeitigen Auftretens von Objekten bzw. Ereignissen untersucht.\parencite [s.][S.110]{Datawarehouse}
Ein Beispiel für solche Zusammenhänge wäre:

Wenn Windeln gekauft werden, wird auch Bier gekauft.

Um die abgeleiteten Assoziationsregeln zu bewerten und Aussagen über diese treffen zu können, werden zwei probabilistische Messwerte herangezogen: Support und Konfidenz.

\paragraph{Support}
Der Support weist auf die Bedeutung von Elementen in einer Menge hin, je höher der Support desto höher die Bedeutung der Elemente.
\parencite [s.][S.110]{Datawarehouse}

\paragraph{Konfidenz}
Mit der Konfidenz wird die Stärke einer Regel ausgedrückt. Auch bei der Konfidenz ist ein hoher Wert wünschenswert, sie beschreibt salopp formuliert die 'Treffsicherheit' einer Assoziationsregel.
\parencite [s.][S.110]{Datawarehouse}

