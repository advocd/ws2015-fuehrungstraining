\chapter{Methode}
Die Datenbasis $D$ ist eine denormalisierte Tabelle, mit einer Menge von Items $I$, die die Spalten der Tabelle darstellen und Transaktionen $\{t_1,t_2,...,t_n\}$, wobei eine Transaktion eine Zeile in der Tabelle ist.
\\
\\
Ein Beispiel für eine Datenbasis sind alle Verkaufstransaktionen innerhalb eines bestimmten Zeitraumes, eine Transaktion ist ein Kund\_inneneinkauf, wobei Items die Artikel im Sortiment sind. 
\parencite[s.][1 Introduction]{IBM}
\\
\\
Wie bereits in der ~\nameref{chap:einleitung} erwähnt werden Assoziationsregeln und Mengen anhand von Support und Konfidenz bewertet.

\section{Support}
Der Support kann einerseits für Mengen von Items berechnet werden, andererseits aber auch für Assoziationsregeln selber.
\subsection{Support einer Menge von Items}
Der Support gibt Auskunft über die Wichtigkeit einer Menge von Items bzw. die Wahrscheinlichkeit mit der die Itemmenge in einer Transaktion vorkommt.
\\
\\
Der Support einer Teilmenge $X=\{A_1, A_2,...,A_n\}$ wird wie folgt berechnet:\\
\begin{equation}
Support(X) = P(A1, A2, ... , An)= \frac{\mid \{t \in D \mid  X \subseteq  t\}\mid}{\mid D \mid}
\end{equation}

Beispiel: 
Der Support der Itemmenge \{Milch\}, ist der Prozentsatz der Kund\_innen die während ihres Einkaufs unter anderem Milch gekauft haben.\parencite[s.][S. 171f]{TU_Dortmund}

\subsection{Support einer Assoziationsregel}
Der Support einer Assoziationsregel $X \rightarrow Y$ ist definiert als der Support von Prämisse/Grundlage($X$) und Konklusion/Folgerung($Y$). Er gibt die relative Häufigkeit an mit der die Regeln in der Datenbasis vorkommt.
Der Support einer Assoziationsregel $X \rightarrow Y$, wobei $X=\{A_1,A_2,...,A_n\}$ und $Y=\{B_1,B_2,...,B_n\}$ wird wie folgt berechnet:\\
\begin{eqnarray}
Support(X \rightarrow Y)= Support(X \cup Y) =  \\
P(A_1, A_2, ... , A_n, B_1, B_2, ..., B_n) = \frac{\mid \{t \in D \mid  X \cup Y \subseteq  t\}\mid}{\mid D \mid}
\end{eqnarray}
\parencite[s.][S. 173]{TU_Dortmund}

\section{Konfidenz}
Die Konfidenz einer Assoziationsregel  $X \rightarrow Y$  wobei  $X=\{A_1,A_2,...,A_n\}$ und $Y=\{B_1,B_2,...,B_n\}$, entspricht der Wahrscheinlichkeit der Konklusion/Folgerung ($Y$) unter Bedingung der Prämisse/Grundlage ($X$). Das bedeutet die Konfidenz einer Assoziationsregel ist der (relative) Anteil der Transaktionen die sowohl X, wie auch Y enthalten. Sie misst sozusagen die Treffsicherheit einer Assoziationsregel.
\\
Die Konfidenz einer Assoziationsregel wird wie folgt berechnet:
\begin{equation}
Confidence(X \rightarrow Y) = Support(X \cup Y)/ Support(X)
\end{equation}
\parencite[s.][S. 173f]{TU_Dortmund}
\newpage

\section{Algorithmus}
Der Apriori Algorithmus arbeitet in zwei Schritten, 1. dem Finden häufiger Mengen, und 2. der Erzeugung von Assoziationsregeln, beide verwenden die Subroutine Apriori Gen.
\\
\\
\begin{algorithm}[H]
\label{alg:apriori}
	\KwIn{$minsupp$, $minconf$}
	\KwOut{Assoziationsregeln}
	1. Schritt: Finden häufiger Mengen\;
	2. Schritt: Erzeugung von Assoziationsregeln\;
\caption{Apriori Algorithmus \parencite[s.][S. 93]{Business_Intelligence}}
\end{algorithm}
\bigskip


\subsection{1. Schritt:  Finden häufiger Mengen}

Die Idee des Apriori Algorithmus (Algorithmus \ref{alg:schritt1}) basiert darauf, dass alle Teilmengen einer häufigen Teilmenge ebenfalls häufig sind, und alle Obermengen einer nicht-häufigen Itemmenge (Teilmenge) ebenfalls nicht häufig sind. 
Das bedeutet, dass Kandidatenmengen mit $k$ Items durch Zusammenführung von Itemmengen mit $k-1$ Items generiert werden könnnen. Alle jene, die Teilmengen, die nicht häufig sind, enthalten werden gelöscht. Häufige Mengen sind nur solche, die genügend hohen Support (mindestens $minsupp$) haben. 
\parencite[s.][2 Discovering Large Itemsets
]{IBM}
\parencite[s.][S. 175]{TU_Dortmund}
\begin{equation}
I_1 \subseteq I_2 \ \mbox{impliziert} \ Support(I2) \leq Support(I_1)
\end{equation}
\bigskip
\bigskip

\begin{algorithm}[H]
\label{alg:schritt1}
	\KwIn{Datenbasis mit Transaktionen, $minsupp$}
	\KwOut{häufige Itemmengen}
	generiere häufige Itemmengen $L_1$ mit $k=1$\;
	\While{solange neue häufigen Mengen gefunden werden}{
		generiere Kandidatenmengen $C_k$ mit ~\nameref{alg:apriorigen} \;
		berechne den Support der Kandidatenmengen\;
		Aufnahme der Mengen mit genügend hohem Support in $L_k$\;
		k++;
	}
	Ausgabe der häufigen Mengen\;	
\caption{Bestimmen häufiger Itemmengen mit Support $\geq minsupp$ \parencite[s.][S. 176]{TU_Dortmund}, \parencite[s.][2.1 Algorithm Apriori]{IBM}}
\end{algorithm}


\subsection{AprioriGen}
AprioriGen (Algorithmus \ref{alg:apriorigen}) ist eine Subroutine des Apriori Algorithmus (Algorithmus \ref{alg:apriori}) und wird sowohl zur Berechnung häufiger Mengen, wie auch zur Erzeugung von Assoziationsregeln verwendet.
\\
Der AprioriGen Algorithmus nutzt aus, dass jede $k-1$ Teilmenge einer Menge aus $L_k$ in $L_{k-1}$ enthalten sein muss. 
\\
\\
\begin{algorithm}[H]
\label{alg:apriorigen}
	\KwIn{Menge häufiger $k-1$ Itemmengen $L_{k-1}$}
	\KwOut{Obermenge $C_k$ der Menge $L_k$}
	1. je zwei häufige $L_{k-1}$ werden mit gleichen ersten ($k-2$) Elementen werden zu einer Kandidatenmenge $C_k$ vereinigt\;
	2. entferne aus $C_k$ diejenigen Itemmengen, deren ($k-1$) Teilmengen nicht alle in $L_{k-1}$ liegen\;
\caption{AprioriGen, \parencite[s.][2.1.1 Apriori Candidate Generation
]{IBM}, \parencite[s.][S. 177f]{TU_Dortmund}}
\end{algorithm}
\newpage
\subsection{2. Schritt: Erzeugung von Assoziationsregeln}
In diesem Schritt des Algorithmus werden nur Itemmengen berücksichtigt die bereits häufig sind, diese wurden in Schritt 1 des Algorithmus berechnet. Es müssen aus den häufigen Itemmengen die gesuchten Assoziationsregeln mit einer Konfidenz $\geq minconf$ bestimmt werden.\\
\\
Dazu wird folgender Zusammenhang ausgenutzt: \\
Wenn für $X,Y$ mit $Y \subseteq X$ die Konfidenz von $(X-Y) \rightarrow Y \geq minconf$, so gilt dies auch für jede Regel $(X-Y') \rightarrow Y$ mit $Y' \subseteq Y$
\\ 
\\
Dementsprechend werden zunächst Assoziationsregeln $X \rightarrow Y $ mit möglichst kurzer Konklusion/Folgerung ($Y$) gebildet, diese wird dann schrittweise erweitert. Es gilt also für Itemmengen $X, Y, Y'$ mit $Y' \subseteq Y \subset X$:
\begin{equation}
	confidence((X-Y') \rightarrow Y') \geq confidence((X - Y) \rightarrow Y)
\end{equation}



\begin{algorithm} [H]
\label{alg:erzeugen_assozationsregeln}
\KwIn{häufige Itemmenge $X$, $minconf$}
\KwOut{Assoziationsregeln für $X$}
Berechne alle Assoziationsregeln $H_1$ mit ausreichender Konfidenz ($confidence \geq minconf$) für $X$\;
sei $H_m$ eine Menge von Konklusionen mit $m$ Items von $X$ so:
	$H_{m+1} = ~\nameref{alg:apriorigen}(H_{m})$\;
\ForAll {Konklusionen $h_{m+1} \in H_{m+1}$} {
überprüfe Konfidenz von $(X-h_{m+1}) \rightarrow h_{m+1}$\;
ist $(confidence(h_{m+1}) \leq minconf$ entferne $h_{m+1}$ aus $H_{m+1}$\;
}
\caption{Erzeugung von Assoziationsregeln aus einer häufigen Itemmenge $X$}
\end{algorithm}
\bigskip
\parencite[s.][S. 179ff]{TU_Dortmund}
