\chapter{Hauptteil}
\label{chap:hauptteil}

\begin{itemize}
\item Hauptteil: Beschreiben uns Analysieren einer Führungssituation (+/-)
\begin{itemize}
\item Welche Kriterien wurden beobachtet?
\item Grad der Kriterien und \ac{AO} - und \ac{BO}
\item ihre Wirksamkeit herausarbeiten = Bewerten
\end{itemize}
\end{itemize}


\section{Fallbeschreibung: Führungssituation}
Tagesbesprechung zwischen meiner \ac{MA}, unserer \ac{FK} und mir. 
Die Wochen davor wurde meine \ac{MA} von unserem \ac{FK} gebeten, ein für Sie fremdes und komplexes Thema in Ihren täglichen Aufgabenbereich zu übernehmen. 
Für diesen Zweck wurde weder eine kurze Einführung oder Einschulung gegeben.
Als die Tagesordnung an dem Punkt angelangt, der auch Ihren neuen Aufgabenbereich betrifft, stellt unsere \ac{FK} fest, dass aufgrund fehlenden Wissens meiner \ac{MA}, Fehler im System sind. 
Was einen emotionalen und persönlichen Gespräch zwischen unserer \ac{FK} und meiner \ac{MA} an Ort und Stelle zur Folge hatte. 
Während diesem Gespräch wurde offensichtlich, dass meine \ac{MA} sich ohne weitere Ressourcen nicht der Aufgabe stellen kann.

\section{Fallanalyse: Führungssituation}
\ac{AO}: Fehlende Unterstützung des MA durch Ressourcen.\\
\\
\ac{BO}: Fehlende Abklärung des Fortschrittes der Lernkurve sowie der Eignung des \ac{MA}

\subsection*{Mögliche Reaktionen:}

\ac{MA} ist frustriert, da Sie durch die persönliche Konfrontation vor anderen \ac{MA} gekränkt wurde. \\
\\
Schlechtes Arbeitsklima durch sofortige Diskussion des Fehlers
Ungünstige Ausgangslage für weiteres Meeting. \\
\\
Betroffenheit und Unsicherheit des anwesenden (unbeteiligten) \ac{MA} 

