\chapter{Fallbeschreibung und Analyse der Führungssituation}
\label{chap:hauptteil}

\section{Rahmen der Situation}
Folgende Führungssituation ist am ehesten mit einer Art Tagesbesprechung zu vergleichen.
Diese Besprechungen laufen so ab, dass die \ac{FK} jeden Morgen, ca. eine Stunde nach Arbeitsbeginn, von Schreibtisch zu Schreibtisch geht und ein kurzes Gespräch   mit der/dem jeweiligen \ac{MA} über die aktuelle Situation führt. 
In einzelnen Fällen kann es vorkommen das der zeitliche Rahmen von ca. 15 Minuten überschritten wird und/oder weitere \ac{MA}'s hinzugezogen werden.
Dabei sind für die Gespräche keine dedizierten Termine geplant, wodurch gezwungener Maßen, dass Gespräch während den eigentlichen operativen Tätigkeiten des/der \ac{MA} spontan integriert werden muss. 
Der Punkt das es sich bei dem Büro um einen großen Raum handelt in dem alle \ac{MA}'s ihren Schreibtisch haben führt dazu das diese Besprechungen nicht in einem diskreten Rahmen ablaufen können.

\pagebreak

\section{Hintergrundinformation zur Führungssituation}

\paragraph*{Mein Persönlicher Hintergrund}
Zu dem Zeitpunkt an dem Situation stattfand war ich noch nicht sehr lange in dem Unternehmen tätig (ca. einen Monate) und in der Position als Praktikant für ein  Projekt angestellt. 
Aufgrund der Unternehmensgröße ist ausschließlich der Geschäftsführer als \ac{FK} vorgesehen und war des weiteren auch mein Projektbetreuer. 
Dabei hatte ich anfangs freie Hand bei der Planung und Realisierung des Projektes 

\paragraph*{Hintergrund meiner \ac{MA}}
\label{para:hintergrungMA}
Ähnlich wie ich, war meine \ac{MA} zum Zeitpunkt der Führungssituation erst seit kurzem in dem Unternehmen und war als Quereinsteigerin selbst noch in der  Einarbeitungsphase. 
Ihre Hauptaufgabe bestand darin, sich um die Buchhaltung und das Tagesgeschäft zu kümmern.
Die Wochen davor wurde meine \ac{MA} von unserem \ac{FK} gebeten, ein für Sie fremdes und komplexes Thema in Ihren täglichen Aufgabenbereich zu übernehmen.\footnote{Sie sollte das rudimentäre firmeneigene Intranet mit neuen Features versehen und zu einer produktiven internen Kommunikationsplattform ausbauen.} 
Für diese Aufgabe wurde weder eine kurze Einführung noch Einschulung gegeben.
\begin{comment}
Tagesbesprechung zwischen meiner \ac{MA}\footnote{Diese wurde erst nachträglich zum Gespräch beordert}, unserer \ac{FK} und mir. 

(siehe \cite{Wunderer2011})
\end{comment}


\section{Ablauf der Führungssituation}
\label{subsec:ablauf_situation}
Die Besprechung begann mit einem direkten Einstieg, bei dem sich meine \ac{FK} nach dem aktuellen Status meines Hauptprojektes erkundigte, mit dem Hinweis das "`wir schon zu sehr dem Zeitplan hinterher hängen"'.
Nachdem ich ein relativ neuer \ac{MA} in dem Unternehmen war, wollte ich an dieser Stelle noch einmal abklären, wie ich korrekt meine Arbeitsschritte in dem Intranet dokumentieren soll, das zum aktuellen Zeitpunkt noch eine "`Baustelle"' ist.
Darauf hin wirkte die \ac{FK} verwundert und teilte mir mit das dies doch problemlos möglich sei und wollte mir dies an Ort und Stelle demonstrieren. 
Dabei stellte die \ac{FK} fest, das dies nicht möglich war und zog sofortig die damit beauftragte \ac{MA} (siehe Abschnitt: \nameref{para:hintergrungMA}) hinzu.
Was einen emotionalen und persönlichen Gespräch zwischen unserer \ac{FK} und meiner \ac{MA} an Ort und Stelle zur Folge hatte. 
Während diesem Gespräch stellte sich heraus, dass meine \ac{MA} sich ohne weitere Ressourcen und Wissen nicht Ihrer Aufgabe stellen kann. 
Dies führte für mich zu einer sehr unangenehme und beklemmende Gesprächsatmosphäre.
Aufgrund der fortgeschrittenen Zeit musste die \ac{FK} zu einer weiteren Besprechung wodurch nicht alle Punkte besprochen werden konnten.

\pagebreak

\section{Fallanalyse: Führungssituation}
\label{sec:fallanalyse}
Die Aufgabe dieses Abschnitts besteht darin, die beschriebene Führungssituation (siehe Abschnitt: \nameref{subsec:ablauf_situation}) mithilfe der in der Lehrveranstaltung (siehe: \cite{Duden2015}) vorgestellten Methode 
 in \ac{AO}- und \ac{BO}-Führungsverhalten zu unterteilen.\\
\\
Aus meiner Sicht stechen für drei Punkte heraus die sich negativ auf die Situation auswirken:

\begin{enumerate}
\item \ac{AO} - nicht ausreichende Abklärung der Eignung sowie der Unterstützung des \ac{MA} durch Ressourcen
\item \ac{AO} - nicht ausreichendes Zeitmanagement  - Es konnten nicht alle offenen Punkte des Meetings abgehalten werden
\item \ac{BO} - fehlende Diskretion im Umgang mit Fehlern
\end{enumerate}





