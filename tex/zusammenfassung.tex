\chapter{Zusammenfassung}
\label{chap:zusammenfassung}

In diesem Kapitel werden zum einen "`\nameref{sec:moegl_reaktionen}"' besprochen welche durch die im zweiten Kapitel analysierten Punkte (siehe Abschnitt: \nameref{sec:fallanalyse}) entstehen können. 
Zum anderen sollen \nameref{sec:handlungsalternativen} aufgezeigt werden, welche aus Sicht des Autors in dieser Situation hilfreich gewesen wären. 
Dabei sollen die \nameref{sec:handlungsalternativen} nicht als allgemeingültige Lösungen sondern vielmehr als Vorschläge anzusehen sein. 

\section{Mögliche Reaktionen}
\label{sec:moegl_reaktionen}

Die hier aufgezählten Kritikpunkte beziehen sich auf die Analyse aus dem zweiten Kapitel (siehe Abschnitt: \nameref{sec:fallanalyse}).
Bei den hier aufgezählten Kritikpunkten handelt es sich ausdrücklich um den subjektiven Versuch des Autors Reaktionen aufzuzeigen, die durch das dokumentierte Führungsverhalten (siehe Abschnitt: \nameref{subsec:ablauf_situation}) entstehen können.

\paragraph*{erster Kritikpunkt}
\label{par:1kritik}
Durch die fehlende oder nicht ausreichende Überprüfung des \ac{MA}'s kann Möglicherweise eine Überforderung der/des \ac{MA} auftreten. 
Auf diese Situation wirkt verstärkend der Faktor ein, dass die/der \ac{MA} nicht ausreichend mit Schulungsmaterial bzw. eigenen Ressourcen, wie beispielsweise Zeit ausgestattet wurde.  

\pagebreak

\paragraph*{zweiter Kritikpunkt}
\label{par:2kritik}
Durch das "`abbrechen"' bzw. nicht vollständig durchführen des Meeting kann es sein das sich der \ac{MA} nicht Wertgeschätzt fühlt. 
Des weiteren besteht die Gefahr, dass dies zu einem klassischen Informationsverlust führen kann. 


\paragraph*{dritter Kritikpunkt} 
\label{par:3kritik}
\ac{MA} ist frustriert, da Sie durch die persönliche Konfrontation vor anderen \ac{MA} gekränkt wurde. 
Was zu einem schlechten Kommunikationsklima aufgrund von Betroffenheit und Unsicherheit der anwesenden (unbeteiligten) \ac{MA} führt. 
Was wiederum eine ungünstige Ausgangslage für das weitere Meeting ist.\\

Zu allen drei Punkten kann ergänzend gesagt werden, dass sich diese Probleme zu einer Resignation oder Frustration des \ac{MA}'s entwickeln können, was sich deutlich auf die Leistungskurve, das Wohlbefinden sowie der Beziehung zur \ac{FK} niederschlagen kann (Stichwort: innere Kündigung).\footnote{je nach eventueller Aufarbeitung/Nachbesprechung durch den \ac{FK} }

\section{Handlungsalternativen}
\label{sec:handlungsalternativen}

Folgende Aufzählung währen denkbare Alternativen für ein angemessenes und zielführendes Führungsverhalten das ein Augenmerk auf Nachhaltigkeit sowie wohl befinden der/des \ac{MA} legt:\\
\\
\ac{AO}: Bereitstellung von Ressourcen (Zeit, Schulungsmaterial, evtl. Übungen) zur Einarbeitung in neuen Themenbereich \\
\\
\ac{AO}: Zeitplan für Übernahme des neuen Tätigkeitsbereichs definieren sowie priorisieren.\\
\\
\ac{BO}: Rücksprache mit \ac{MA} nach aktuellen Status des Lernfortschritts oder 
auftretenden Problemen \\
\\
\ac{BO}: Besprechung des Problems auf einen späteren vier-Augen-Termin verschieben\\
\\

Wie der Abschnitt \nameref{sec:moegl_reaktionen} aufzeigt liegen mögliche (interpretierten) Defizite in der Wertschätzung der \ac{MA} welche durch diverse Möglichkeiten gelöst werden könnten. 
Dazu ist ergänzend zu erwähnen, solche Intentionen schon Ansatzweise etabliert wurden da die \ac{MA} in dem Unternehmen, in dem die  geschilderten Situation stattgefunden hat, eine große Eigenverantwortung sowie einen großzügigen Handlungsfreiraum in Planung und Umsetzung ihre Projekte haben.
Dies wird leider nicht konsequent durchgezogen da Schlussendlich die persönliche Meinung der \ac{FK} ausschlaggebend ist.\\
\\
Eine sehr interessant Möglichkeit besteht in dem "`Empowerment"'der/des \ac{MA}.
Drucker \& Ferber sagen: 

\begin{quote}
"`Nur durch richtige Personalentscheidung kann wirtschaftlich erfolgreich gehandelt werden. Empowerment schließt Verantwortung und Vertrauen sowohl in Menschen als auch in Organisationen mit ein"'\footnote{vgl.: \cite{Drucker2009}}
\end{quote}


Denkbare wäre es den schon vorhandenen Ansatz das \ac{MA} eine große Verantwortung bzw. Mitspracherecht haben konsequenter umzusetzen. 
Dies muss nicht gezwungener Maßen in jedem einzelnen Projekt so realisiert werden. 
In einer Testphase könnte evaluiert werden ob bei Nicht-kritischen Projekten oder Teilaufgaben die Entscheidungsgewalt an die jeweilige/n \ac{MA} delegiert wird. 
Sollte dies aus diversen Gründen nicht möglich sein, so sollte dies mindestens den \ac{MA} klar kommuniziert werden und diese um beispielsweise "`Entscheidungsgrundlagen"' gebeten werden. \\
\\
Selbstverständlich ist dieser Führungsstil durchaus mit Risiken verbunden, allerdings sollten wir uns als mündige Führungskräfte durchaus damit auseinander setzen wenn wir, gemeinsam mit unseren verantwortungsvollen Mitarbeiter\_innen, erfolgreich die Zukunft gestalten wollen.
 

